\begin{appendix}
\chapter{Units and constants}

The basic units are cm for distances, g for (macroscopic) masses, 
and ns for times. 
Particle energies, momenta and masses are expressed 
in eV, eV/\(c\) and eV/\(c^{2}\), respectively.
For example, the electron mass is given in eV/\(c^{2}\), 
whereas the mass density of a material is given in g/cm\(^{3}\). 
The mass of an atom is specified in terms of the atomic mass number \(A\).
  
There are a few exceptions from this system of units, though.
\begin{itemize}
  \item
  The unit for the magnetic field \(\mathbf{B}\) corresponding
  to the above system of units (10\(^{-5}\) Tesla) is impractical. 
  Instead, magnetic fields are expressed in Tesla.
  \item
  Pressures are specified in Torr.
  \item
  Electric charge is expressed in fC. 
\end{itemize}
A summary of commonly used quantities and their units is given 
in Table~\ref{Tab:Units}.
\begin{table}
  \centering
  \begin{tabular}{l l}
    \toprule
    physical quantity & unit \\
    \midrule
    length & cm \\
    mass & g \\
    time & ns \\
    temperature & K \\
    electric potential & V \\
    electric charge & fC \\
    \midrule
    energy & eV \\
    pressure & Torr \\
    electric field & V / cm \\
    magnetic field & Tesla \\
    electric current & fC / ns \\
    angle & rad \\
    \bottomrule
  \end{tabular}
  \caption{Physical units.}
  \label{Tab:Units}
\end{table}
  
The values of the physical constants used in the code are defined  
in the file \texttt{FundamentalConstants.hh}.

\chapter{Gases}

Table \ref{Tab:GasesMagboltz11} shows a list of the gases 
available in the current version of \textsc{Magboltz}. 
The star rating represents an estimate of the reliability 
of the cross-section data for the respective gas. 
A rating of ``5*'' indicates a detailed, well-validated 
description of the cross-sections, while 
``2*'' indicates a low quality, that is 
a coarse modelling of the cross-sections 
associated with large uncertainties.

\begin{longtable}{l l l}
  \toprule
   chem. symbol & name & rating \\
  \midrule
  \(^{4}\)He & helium   & 5* \\
  \(^{3}\)He & helium-3 & 5* \\
          Ne & neon     & 5* \\
          Ar & argon    & 5* \\
          Kr & krypton  & 4* \\
          Xe & xenon    & 4* \\
  \midrule
  Cs         & cesium   & 2* \\
  Hg         & mercury  & 2* \\
  \midrule
  H\(_{2}\) & hydrogen  & 5* \\
  para H\(_{2}\) & para hydrogen  & 5* \\
  D\(_{2}\) & deuterium & 5* \\
  ortho D\(_{2}\) & ortho deuterium & 4* \\
  N\(_{2}\) & nitrogen  & 5* \\
  O\(_{2}\) & oxygen    & 5* \\
  F\(_{2}\) & fluorine  & 2* \\
  \midrule
  CO         & carbon monoxide & 5* \\
  NO         & nitric oxide    & 2* \\
  \midrule
  H\(_{2}\)O & water           & 5* \\ 
  CO\(_{2}\) & carbon dioxide  & 5* \\
  N\(_{2}\)O & nitrous oxide   & 4* \\
  O\(_{3}\)  & ozone           & 3* \\
  H\(_{2}\)S & hydrogen sulfide & 2* \\
  COS        & carbonyl sulfide & 2* \\
  CS\(_{2}\) & carbon disulfide & 2* \\
  \midrule
  CH\(_{4}\)           & methane   & 5* \\
  CD\(_{4}\)           & deuterated methane & 4* \\
  C\(_{2}\)H\(_{6}\)   & ethane    & 5* \\
  C\(_{3}\)H\(_{8}\)   & propane   & 4* \\
  nC\(_{4}\)H\(_{10}\) & n-butane  & 4* \\
  iC\(_{4}\)H\(_{10}\) & isobutane & 4* \\ 
  nC\(_{5}\)H\(_{12}\) & n-pentane & 4* \\  
  neo-C\(_{5}\)H\(_{12}\) & neopentane & 4* \\
  C\(_{2}\)H\(_{4}\)   & ethene    & 4* \\
  C\(_{2}\)H\(_{2}\)   & acetylene & 4* \\
  C\(_{3}\)H\(_{6}\)   & propene   & 4* \\
  cC\(_{3}\)H\(_{6}\)   & cyclopropane & 4* \\
  \midrule
  CH\(_{3}\)OH         & methanol        & 4* \\
  C\(_{2}\)H\(_{5}\)OH & ethanol         & 4* \\
  C\(_{3}\)H\(_{7}\)OH & isopropanol     & 3* \\
  nC\(_{3}\)H\(_{7}\)OH & n-propanol     & 3* \\
  C\(_{3}\)H\(_{8}\)O\(_{2}\)  & methylal & 2* \\
  C\(_{4}\)H\(_{10}\)O\(_{2}\) & DME      & 5* \\
  \midrule
  CF\(_{4}\)           & tetrafluoromethane & 5* \\
  CHF\(_{3}\)          & fluoroform         & 3* \\
  C\(_{2}\)F\(_{6}\)   & hexafluoroethane   & 4* \\
  C\(_{2}\)H\(_{2}\)F\(_{4}\) & tetrafluoroethane & 3* \\
  C\(_{3}\)F\(_{8}\)   & octafluoropropane  & 4* \\
  SF\(_{6}\)           & sulfur hexafluoride  & 4* \\
  BF\(_{3}\)           & boron trifluoride    & 4* \\ 
  CF\(_{3}\)Br         & bromotrifluoromethane & 3* \\
  \midrule
  NH\(_{3}\)            & ammonia  & 4* \\
  N(CH\(_{3}\))\(_{3}\) & TMA      & 3* \\
  SiH\(_{4}\)           & silane   & 3* \\
  GeH\(_{4}\)           & germane  & 4* \\
  \bottomrule
  \caption{Gases available in \textsc{Magboltz} 11.13.} 
  \label{Tab:GasesMagboltz11}
\end{longtable}

\chapter{Solids}\label{Sec:Solids}
The constructors of the \texttt{Solid} classes usually come in two versions:
one where the solid is kept in its default orientation 
(\ie the local frame of the solid and the global frame are identical 
except for a translation), and one where the solid's orientation is set 
explicitly. This is done by specifying the direction of the solid's local 
$z$ axis in the global frame
\section{Box}
\texttt{SolidBox} describes a rectangular cuboid (Fig.~\ref{Fig:SolidBox}).
The simplest version of the constructor creates a box the edges of which 
are aligned with the axes of the coordinate system. 
\begin{lstlisting}
SolidBox(const double cx, const double cy, const double cz, 
         const double lx, const double ly, const double lz);
\end{lstlisting}
\begin{description}
\item[cx,cy,cz] Coordinates of the centre of gravity of the box.
\item[lx,ly,lz] Half-widths of the box along $x$, $y$, and $z$.
\end{description}
A box with a non-default orientation in space is created using
\begin{lstlisting}
SolidBox(const double cx, const double cy, const double cz, 
         const double lx, const double ly, const double lz, 
         const double dx, const double dy, const double dz);
\end{lstlisting} 

\tdplotsetmaincoords{60}{120}
\begin{figure}
\centering
\begin{tikzpicture}[panel/.style={fill=lightgray,fill opacity=0.2},line join=round,>=stealth,tdplot_main_coords]
\pgfmathsetmacro{\lx}{1.5}
\pgfmathsetmacro{\ly}{3}
\pgfmathsetmacro{\lz}{2}
\draw[thick,blue,dashed] (-\lx,0,0) -- (\lx,0,0);
\draw[thick,blue,dashed] (0,-\ly,0) -- (0,\ly,0);
\draw[thick,blue,dashed] (0,0,-\lz) -- (0,0,\lz);
\draw[thick,blue,->] (\lx,0,0) -- (5,0,0) node[below left]{$x$};
\draw[thick,blue,->] (0,\ly,0) -- (0,5,0) node[right]{$y$};
\draw[thick,blue,->] (0,0,\lz) -- (0,0,5) node[above]{$z$};

\filldraw[panel,thick] (-\lx,\ly,\lz) -- (\lx,\ly,\lz) -- (\lx,-\ly,\lz) -- (-\lx,-\ly,\lz) -- cycle;
\fill[panel] (-\lx,\ly,-\lz) -- (\lx,\ly,-\lz) -- (\lx,-\ly,-\lz) -- (-\lx,-\ly,-\lz) -- cycle; 
\filldraw[panel,thick] (-\lx,\ly,\lz) -- (\lx,\ly,\lz) -- (\lx,\ly,-\lz) -- (-\lx,\ly,-\lz) -- cycle; 
\fill[panel] (-\lx,-\ly,\lz) -- (\lx,-\ly,\lz) -- (\lx,-\ly,-\lz) -- (-\lx,-\ly,-\lz) -- cycle; 
\fill[panel] (-\lx,-\ly,\lz) -- (-\lx,\ly,\lz) -- (-\lx,\ly,-\lz) -- (-\lx,-\ly,-\lz) -- cycle; 
\filldraw[panel,thick] (\lx,-\ly,\lz) -- (\lx,\ly,\lz) -- (\lx,\ly,-\lz) -- (\lx,-\ly,-\lz) -- cycle;
\draw[line join=round,thick,dashed] (\lx,-\ly,-\lz) -- (-\lx,-\ly,-\lz) -- (-\lx,-\ly,\lz);
\draw[line join=round,thick,dashed] (-\lx,-\ly,-\lz) -- (-\lx,\ly,-\lz);

\draw[<->] (\lx,0,-\lz) -- (\lx,0,0) node[midway,left]{$l_{z}$}; 
\draw[<->] (\lx,0,\lz) -- (\lx,0,0) node[midway,left]{$l_{z}$};

\draw[<->] (-\lx,\ly,0) -- (0,\ly,0) node[midway,above]{$l_{x}$}; 
\draw[<->] (\lx,\ly,0) -- (0,\ly,0) node[midway,above]{$l_{x}$};

\draw[<->] (0,-\ly,\lz) -- (0,0,\lz) node[midway,below]{$l_{y}$}; 
\draw[<->] (0,\ly,\lz) -- (0,0,\lz) node[midway,below]{$l_{y}$};
\end{tikzpicture}
\caption{\texttt{SolidBox} centred at $\left(0,0,0\right)$, defined 
by its half-widths $l_{x}$, $l_{y}$, $l_{z}$.}
\label{Fig:SolidBox}
\end{figure}

\section{Tube}
\texttt{SolidTube} describes a cylinder. The simplest constructor requires
the location of the centre, the (outer) radius and the half-length. 
\begin{lstlisting}
SolidTube(const double cx, const double cy, const double cz, const double r,
          const double lz);
\end{lstlisting}
\begin{description}
\item[cx,cy,cz] Coordinates of the centre of gravity of the cylinder.
\item[r] Outer radius of the cylinder.
\item[lz] Half-length of the cylinder.
\end{description}
By default, the central axis of the cylinder is collinear with the $z$-axis,
 as illustrated in Fig.~\ref{Fig:SolidTube}. 
To create a cylinder with a different orientation, the constructor
\begin{lstlisting}
SolidTube(const double cx, const double cy, const double cz, const double r,
          const double lz, const double dx, const double dy, const double dz);
\end{lstlisting} 
can be used. 
Constructors for cylinders with non-zero inner radius are also available.

\begin{figure}
\centering
\tdplotsetmaincoords{60}{120}
\begin{tikzpicture}[panel/.style={fill=lightgray,fill opacity=0.2},line join=round,>=stealth,tdplot_main_coords]
\pgfmathsetmacro{\rmax}{2}
\pgfmathsetmacro{\rmin}{0.4}
\pgfmathsetmacro{\lz}{3}

\begin{scope}[canvas is xy plane at z=0]
  \coordinate (MLZ) at (\tdplotmainphi-60:\rmax);
  \draw[thin] (\tdplotmainphi:\rmax) arc (\tdplotmainphi:\tdplotmainphi-180:\rmax);
  \draw[thin,dashed] (\tdplotmainphi:\rmax) arc (\tdplotmainphi:\tdplotmainphi+180:\rmax);
  \draw[thin,dashed] (0,0) circle (\rmin);
\end{scope}
\begin{scope}[canvas is xy plane at z=-\lz]
  \coordinate (BR) at (\tdplotmainphi:\rmax);
  \coordinate (BL) at (\tdplotmainphi-180:\rmax);
  \coordinate (BLZ) at (\tdplotmainphi-60:\rmax);
  \draw[dashed] (\tdplotmainphi:\rmax) arc (\tdplotmainphi:\tdplotmainphi+180:\rmax);
  \draw[thin,dashed] (0,0) circle (\rmin);
  \draw[<->,thick,solid] (0,0,0) -- (0:\rmax) node[midway,above]{$r$};
\end{scope}
\begin{scope}[canvas is xy plane at z=\lz]
  \coordinate (AZ) at (\tdplotmainphi-90:\rmin);
  \coordinate (TLZ) at (\tdplotmainphi-60:\rmax);
  \filldraw[panel,thick] (BR) arc (\tdplotmainphi:\tdplotmainphi-180:\rmax) -- (\tdplotmainphi-180:\rmax) arc (\tdplotmainphi-180:\tdplotmainphi:\rmax) --cycle;
 \filldraw[panel,thick] (0,0) circle (\rmax); 
 \filldraw[thick,fill=white] (0,0) circle (\rmin);
 \draw[<->,thick] (MLZ) -- (TLZ) node[midway,right]{$l_{z}$};
 \draw[<->,thick] (MLZ) -- (BLZ) node[midway,right]{$l_{z}$};
\end{scope}

\draw[thick,blue,dashed] (-\rmax,0,0) -- (\rmax,0,0);
\draw[thick,blue,dashed] (0,-\rmax,0) -- (0,\rmax,0);
\draw[thick,blue,dashed] (0,0,-\lz) -- (AZ);
\draw[thick,blue,->] (\rmax,0,0) -- (5,0,0) node[below left]{$x$};
\draw[thick,blue,->] (0,\rmax,0) -- (0,5,0) node[below right]{$y$};
\draw[thick,blue,->] (AZ) -- (0,0,5) node[above]{$z$};

\end{tikzpicture}
\caption{\texttt{SolidTube} centred at $\left(0,0,0\right)$.}
\label{Fig:SolidTube}
\end{figure}

When determining the surface panels (\eg for use in neBEM), 
the cylinder is approximated as a polygon with a finite number of panels.
The type of polygon to be used can be set using
\begin{lstlisting}
void SetSectors(const unsigned int n);
\end{lstlisting}
The number of corners of the polygon equals $4(n - 1)$.
Thus, $n = 2$ will produce a square, $n = 3$ an octagon.
By default, the polygon used for approximating the cylinder 
is inscribed in a circle of the specified radius. 
If the ``average-radius'' flag is activated using
\begin{lstlisting}
void SetAverageRadius(const bool average);
\end{lstlisting}
the radius will be interpreted as the mean radius of the polygon
that approximates the cylinder.
By default, the list of surface panels will include the ``lids'' at 
$\pm z$. This can be switched off using the functions 
\begin{lstlisting}
void SetTopLid(const bool closed);
void SetBottomLid(const bool closed);
\end{lstlisting}

\section{Sphere}
\texttt{SolidSphere} has two constructors.
The first one takes the location of the 
centre and the outer radius of the sphere.
\begin{lstlisting}
SolidSphere(const double cx, const double cy, const double cz,
            const double r);
\end{lstlisting}
\begin{description}
\item[cx,cy,cz] Coordinates of the centre of the sphere.
\item[r] Radius of the sphere.
\end{description}
The second one takes the coordinates of the centre and the inner and 
outer radii.
\begin{lstlisting}
SolidSphere(const double cx, const double cy, const double cz,
            const double rmin, const double rmax);
\end{lstlisting}

\begin{figure}
\centering
\tdplotsetmaincoords{60}{120}
\begin{tikzpicture}[line join=round,>=stealth,tdplot_main_coords]
\pgfmathsetmacro{\radius}{2}
\draw[thick,blue,dashed] (-\radius,0,0) -- (\radius,0,0);
\draw[thick,blue,dashed] (0,-\radius,0) -- (0,\radius,0);
\draw[thick,blue,dashed] (0,0,-\radius) -- (0,0,\radius);
\draw[thick,blue,->] (\radius,0,0) -- (5,0,0) node[below left]{$x$};
\draw[thick,blue,->] (0,\radius,0) -- (0,5,0) node[below right]{$y$};
\draw[thick,blue,->] (0,0,\radius) -- (0,0,5) node[above]{$z$};
\shadedraw[tdplot_screen_coords,ball color = white,very thin,opacity=0.2] (0,0) circle (\radius);
\draw[thin] (\tdplotmainphi:\radius) arc (\tdplotmainphi:\tdplotmainphi-180:\radius);
\draw[thin,dashed] (\tdplotmainphi:\radius) arc (\tdplotmainphi:\tdplotmainphi+180:\radius);
\tdplotsetthetaplanecoords{0}
\tdplotdrawarc[dashed,thin,tdplot_rotated_coords]{(0,0,0)}{\radius}{\tdplotmaintheta-270}{\tdplotmaintheta-90}{}{}
\tdplotdrawarc[thin,tdplot_rotated_coords]{(0,0,0)}{\radius}{\tdplotmaintheta-90}{\tdplotmaintheta+90}{}{}
\draw[<->,thick,solid] (0,0,0) -- (-45:\radius) node[pos=0.7,above]{$r$};
\end{tikzpicture}
\caption{\texttt{SolidSphere} centred at $\left(0,0,0\right)$.}
\label{Fig:SolidSphere}
\end{figure}

When calculating surface panels (\eg for use in neBEM), 
the sphere is approximated by a set of parallelograms.
The parameter $n$, set using
\begin{lstlisting}
void SetMeridians(const unsigned int n);
\end{lstlisting} 
specifies the number of meridians and also the number of parallels.

\section{Hole}
\texttt{SolidHole} describes an, optionally tapered, cylindrical hole 
in a box. Mandatory parameters are the location of the centre, 
the radii and the dimensions of the box (see Fig.~\ref{Fig:SolidHole} for 
an illustration of the parameters).
\begin{lstlisting}
SolidHole(const double cx, const double cy, const double cz,
          const double rup, const double rlow,
          const double lx, const double ly, const double lz);      
\end{lstlisting}
\begin{description}
  \item[cx,cy,cz] Location of the point that is on the central axis of the hole, half-way between the two planes of the box perpendicular to the central axis of the hole.
  \item[rup,rlow] Radius of the hole as measured at the ``upper'' and at the ``lower'' surface of the box.
  \item[lx,ly,lz] Half-lenghts of the box.
\end{description}
The central axis of the hole is collinear with the $z$-axis. 
To create a hole with a non-default orientation, the constructor
\begin{lstlisting}
SolidHole(const double cx, const double cy, const double cz,
          const double rup, const double rlow,
          const double lx, const double ly, const double lz,
          const double dx, const double dy, const double dz);
\end{lstlisting}
is available.
\begin{figure}
\centering
\tdplotsetmaincoords{60}{120}
\begin{tikzpicture}[panel/.style={fill=lightgray,fill opacity=0.2},line join=round,>=stealth,tdplot_main_coords]
\pgfmathsetmacro{\rtop}{1.5}
\pgfmathsetmacro{\rbot}{1.15}
\pgfmathsetmacro{\lx}{2}
\pgfmathsetmacro{\ly}{2}
\pgfmathsetmacro{\lz}{2}

\filldraw[panel,thick] (-\lx,\ly,\lz) -- (\lx,\ly,\lz) -- (\lx,-\ly,\lz) -- (-\lx,-\ly,\lz) -- cycle;
\fill[panel] (-\lx,\ly,-\lz) -- (\lx,\ly,-\lz) -- (\lx,-\ly,-\lz) -- (-\lx,-\ly,-\lz) -- cycle; 
\filldraw[panel,thick] (-\lx,\ly,\lz) -- (\lx,\ly,\lz) -- (\lx,\ly,-\lz) -- (-\lx,\ly,-\lz) -- cycle; 
\fill[panel] (-\lx,-\ly,\lz) -- (\lx,-\ly,\lz) -- (\lx,-\ly,-\lz) -- (-\lx,-\ly,-\lz) -- cycle; 
\fill[panel] (-\lx,-\ly,\lz) -- (-\lx,\ly,\lz) -- (-\lx,\ly,-\lz) -- (-\lx,-\ly,-\lz) -- cycle; 
\filldraw[panel,thick] (\lx,-\ly,\lz) -- (\lx,\ly,\lz) -- (\lx,\ly,-\lz) -- (\lx,-\ly,-\lz) -- cycle;
\draw[line join=round,thick,dashed] (\lx,-\ly,-\lz) -- (-\lx,-\ly,-\lz) -- (-\lx,-\ly,\lz);
\draw[line join=round,thick,dashed] (-\lx,-\ly,-\lz) -- (-\lx,\ly,-\lz);

\begin{scope}[canvas is xy plane at z=-\lz]
  \coordinate (BR) at (\tdplotmainphi:\rbot);
  \coordinate (BL) at (\tdplotmainphi-180:\rbot);
  \draw[dashed] (0,0) circle (\rbot);
  \draw[<->,thick,solid] (0,0,0) -- (\tdplotmainphi+180:\rbot) node[midway,above]{$r_{\text{low}}$};
\end{scope}
\begin{scope}[canvas is xy plane at z=\lz]
  \coordinate (AZ) at (\tdplotmainphi-90:\rtop);
  \filldraw[thick,fill=white] (0,0) circle (\rtop);
  \draw[dashed] (\tdplotmainphi:\rtop) -- (BR);
  \draw[dashed] (\tdplotmainphi-180:\rtop) -- (BL);
  \draw[<->,thick,solid] (0,0,0) -- (\tdplotmainphi+180:\rtop) node[midway,above]{$r_{\text{up}}$};
\end{scope}

\draw[thick,blue,dashed] (-\lx,0,0) -- (\lx,0,0);
\draw[thick,blue,dashed] (0,-\ly,0) -- (0,\ly,0);
\draw[thick,blue,dashed] (0,0,-\lz) -- (AZ);
\draw[thick,blue,->] (\lx,0,0) -- (5,0,0) node[below left]{$x$};
\draw[thick,blue,->] (0,\ly,0) -- (0,5,0) node[right]{$y$};
\draw[thick,blue,->] (AZ) -- (0,0,5) node[above]{$z$};
\end{tikzpicture}
\caption{\texttt{SolidHole} centred at $\left(0,0,0\right)$.}
\label{Fig:SolidHole}
\end{figure}

Like in \texttt{SolidTube}, the type of polygon used for approximating the 
hole when calculating the surface panels can be specified using 
\begin{lstlisting}
void SetSectors(const unsigned int n);
\end{lstlisting}

\section{Ridge}
\texttt{SolidRidge} describes a ridge, similar to a Toblerone bar. 
The constructor takes the location of the centre and size of the floor, 
and the position of the ridge proper.
\begin{lstlisting}
SolidRidge(const double cx, const double cy, const double cz,
           const double lx, const double ly, const double hz,
           const double hx);
\end{lstlisting} 
\begin{description}
  \item[cx,cy,cz] Centre of the floor of the ridge in the $\left(x,y\right)$ plane at $z=0$.
  \item[lx,ly] Half-lenghts of the floor.
  \item[hz] Height of the ridge measured from the floor.
  \item[hx] Offset in the $x$-direction of the ridge. If the offset is set to 0, then the ridge will be symmetric.  
\end{description}
An illustration is given in Fig.~\ref{Fig:SolidRidge}.
By default, the ridge is taken to be parallel with the $y$-axis.
\begin{figure}
\centering
\tdplotsetmaincoords{60}{160}
\begin{tikzpicture}[panel/.style={fill=lightgray,fill opacity=0.2},line join=round,>=stealth,tdplot_main_coords]
\pgfmathsetmacro{\lx}{2}
\pgfmathsetmacro{\ly}{3}
\pgfmathsetmacro{\hz}{2}
\pgfmathsetmacro{\hx}{0.5}
\pgfmathsetmacro{\zo}{\hz*\lx/(\hx+\lx)}
\draw[thick,blue,dashed] (-\lx,0,0) -- (\lx,0,0);
\draw[thick,blue,dashed] (0,-\ly,0) -- (0,\ly,0);
\draw[thick,blue,dashed] (0,0,0) -- (0,0,\zo);
\draw[thick,blue,->] (\lx,0,0) -- (5,0,0) node[below left]{$x$};
\draw[thick,blue,->] (0,\ly,0) -- (0,5,0) node[right]{$y$};
\draw[thick,blue,->] (0,0,\zo) -- (0,0,5) node[above]{$z$};

\filldraw[panel,dashed] (-\lx,-\ly,0) -- (\lx,-\ly,0) -- (\hx,-\ly,\hz) -- cycle;
\filldraw[panel,thick] (-\lx,\ly,0) -- (\lx,\ly,0) -- (\hx,\ly,\hz) -- cycle;
\filldraw[panel,thick] (-\lx,\ly,0) -- (-\lx,-\ly,0) -- (\hx,-\ly,\hz) -- (\hx,\ly,\hz) -- cycle; 
\filldraw[panel,thick] (\lx,\ly,0) -- (\lx,-\ly,0) -- (\hx,-\ly,\hz) -- (\hx,\ly,\hz) -- cycle; 

\draw[dashed,thin] (\hx,\ly,0) -- (\hx,-\ly,0) -- (\hx,-\ly,\hz);
\draw[<->,thick] (\hx,\ly,0) -- (\hx,\ly,\hz) node[midway,left]{$h_{z}$};
\draw[<->,thick] (\hx,-\ly,0) -- (0,-\ly,0) node[midway,above]{$h_{x}$};
\draw[<->,thick] (-\lx,-\ly,0) -- (-\lx,0,0) node[midway,right]{$l_{y}$}; 
\draw[<->,thick] (-\lx,+\ly,0) -- (-\lx,0,0) node[midway,right]{$l_{y}$}; 
\draw[<->,thick] (-\lx,\ly,0) -- (0,\ly,0) node[midway,below]{$l_{x}$};
\draw[<->,thick] (+\lx,\ly,0) -- (0,\ly,0) node[midway,below]{$l_{x}$};
\end{tikzpicture}
\caption{\texttt{SolidRidge} in its default orientation, centred at 
$\left(0,0,0\right)$.}
\label{Fig:SolidRidge}
\end{figure}

\section{Extrusion}
\texttt{SolidExtrusion} (illustrated in Fig.~\ref{Fig:SolidExtrusion})
describes a volume
generated by extruding a polygon along its normal axis (default: $z$-axis).
The constructor takes the half-length along 
$z$, and the $x, y$ coordinates of the polygon defining the extrusion. 
\begin{lstlisting}
SolidExtrusion(const double lz,
               const std::vector<double>& xp, const std::vector<double>& yp); 
\end{lstlisting} 
\begin{description}
  \item[lz] Half-length of the extrusion along $z$.
  \item[xp,yp] $x,y$ coordinates of the point defining the extrusion profile.
\end{description}
To create an extrusion with an orientation or offset different from the 
default one, the constructor
\begin{lstlisting}
SolidExtrusion(const double lz,
               const std::vector<double>& xp, const std::vector<double>& yp,
               const double cx, const double cy, const double cz,
               const double dx, const double dy, const double dz); 
\end{lstlisting} 
can be used.

\begin{figure}
\centering
\tdplotsetmaincoords{80}{120}
\begin{tikzpicture}[panel/.style={fill=lightgray,fill opacity=0.2},line join=round,>=stealth,tdplot_main_coords]
\pgfmathsetmacro{\xa}{-3}
\pgfmathsetmacro{\ya}{-2}
\pgfmathsetmacro{\xb}{-2}
\pgfmathsetmacro{\yb}{2}
\pgfmathsetmacro{\xc}{2}
\pgfmathsetmacro{\yc}{0}
\pgfmathsetmacro{\xd}{1}
\pgfmathsetmacro{\yd}{-1}

\pgfmathsetmacro{\lx}{\xc}
\pgfmathsetmacro{\ly}{1}
\pgfmathsetmacro{\lz}{2}
\pgfmathsetmacro{\hx}{0.5}
\draw[thick,blue,dashed] (0,0,0) -- (\lx,0,0);
\draw[thick,blue,dashed] (0,0,0) -- (0,\ly,0);
\draw[thick,blue,dashed] (0,0,-\lz) -- (0,0,\lz);
\draw[thick,blue,->] (\lx,0,0) -- (5,0,0) node[below left]{$x$};
\draw[thick,blue,->] (0,\ly,0) -- (0,5,0) node[right]{$y$};
\draw[thick,blue,->] (0,0,\lz) -- (0,0,5) node[above]{$z$};

\filldraw[panel,dashed] (\xa,\ya,-\lz) -- (\xb,\yb,-\lz) -- (\xc,\yc,-\lz) -- (\xd,\yd,-\lz) -- cycle;
\draw[dashed] (\xa,\ya,0) -- (\xb,\yb,0) -- (\xc,\yc,0) -- (\xd,\yd,0) -- cycle;
\filldraw[panel,thick] (\xa,\ya,\lz) -- (\xb,\yb,\lz) -- (\xc,\yc,\lz) -- (\xd,\yd,\lz) -- cycle;
\filldraw[panel,thick] (\xb,\yb,\lz) -- (\xc,\yc,\lz) -- (\xc,\yc,-\lz) -- (\xb,\yb,-\lz) -- cycle;
\filldraw[panel,thick] (\xc,\yc,\lz) -- (\xd,\yd,\lz) -- (\xd,\yd,-\lz) -- (\xc,\yc,-\lz) -- cycle;
\draw[dashed] (\xa,\ya,-\lz) -- (\xa,\ya,\lz);

\draw[<->,thick] (0,\ly,-\lz) -- (0,\ly,0) node[midway,right]{$l_{z}$}; 
\draw[<->,thick] (0,\ly,0) -- (0,\ly,\lz) node[midway,right]{$l_{z}$}; 
\end{tikzpicture}
\caption{\texttt{SolidExtrusion} in its default orientation.}
\label{Fig:SolidExtrusion}
\end{figure}

\end{appendix}

