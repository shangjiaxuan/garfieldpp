\section{Examples}

Section~\ref{Sec:ExampleTube} discusses the 
calculation of transport parameters with \textsc{Magboltz}, 
the use of analytic field calculation techniques, 
``macroscopic'' simulation of electron and ion drift lines, 
and the calculation of induced signals. 
 
Microscopic transport of electrons and 
the use of finite element field maps are dealt with in 
Sec.~\ref{Sec:ExampleGem}. 

Section~\ref{Sec:ExampleSilicon} presents an example of 
the simulation of drift lines and induced signals in a 
silicon sensor.
 
Further examples can be found on the webpage 
(\url{http://garfieldpp.web.cern.ch/garfieldpp/Examples}) and 
in the directory \texttt{Examples} of the source tree.
 
\subsection{Drift tube}\label{Sec:ExampleTube}
In this example, we consider a drift tube with an outer diameter of 
15\,mm and a wire diameter of 50\,\textmu{m}, similar to the 
ATLAS small-diameter muon drift tubes (sMDTs).

\subsubsection{Gas table}
First, we prepare a table of transport parameters 
(drift velocity, diffusion coefficients, Townsend coefficient,
and attachment coefficient) as a function 
of the electric field \(\mathbf{E}\)  
(and, in general, also the magnetic field \(\mathbf{B}\) 
as well as the angle between \(\mathbf{E}\) and \(\mathbf{B}\)).
In this example, we use a gas mixture of 93\% argon and 7\% 
carbon dioxide at a pressure of 3\,atm and room temperature.
\begin{lstlisting}
MediumMagboltz gas;
gas.SetComposition("ar", 93., "co2", 7.);
// Set temperature [K] and pressure [Torr].
gas.SetPressure(3 * 760.);
gas.SetTemperature(293.15);
\end{lstlisting} 
We also have to specify the number of electric fields to be 
included in the table and the electric field range to be covered. 
Here we use 20 field points between 100\,V\,/\,cm and 100\,kV\,/\,cm 
with logarithmic spacing. 
\begin{lstlisting}
gas.SetFieldGrid(100., 100.e3, 20, true);
\end{lstlisting}
Now we run \textsc{Magboltz} to generate the gas table for this grid. 
As input parameter we have to specify the number of collisions 
(in multiples of \(10^{7}\)) over which the electron is traced 
by \textsc{Magboltz}.
\begin{lstlisting}
const int ncoll = 10;
gas.GenerateGasTable(ncoll);
\end{lstlisting}
This calculation will take a while, don't panic. 
After the calculation is finished, we save the gas table to a 
file for later use.
\begin{lstlisting}
gas.WriteGasFile("ar_93_co2_7.gas");
\end{lstlisting}
Once we have saved the transport parameters to file 
we can skip the steps above, 
and simply import the table in our program using
\begin{lstlisting}
gas.LoadGasFile("ar_93_co2_7.gas");
\end{lstlisting} 

In order to make sure the calculation of the gas table was successful, 
it is a good idea to plot, for instance, 
the drift velocity as a function of the electric field.
\begin{lstlisting}
ViewMedium mediumView;
mediumView.SetMedium(&gas);
mediumView.PlotElectronVelocity('e');
\end{lstlisting}
\subsubsection{Electric Field}
For calculating the electric field inside the tube, 
we use the class \texttt{ComponentAnalyticField} which can handle 
(two-dimensional) arrangements of wires, planes and tubes.
\begin{lstlisting}
ComponentAnalyticField cmp;
\end{lstlisting}
We first set the medium inside the active region. 
\begin{lstlisting}
cmp.SetMedium(&gas); 
\end{lstlisting}
Next we add the elements defining the electric field, 
\ie~the wire and the tube.
\begin{lstlisting}
// Wire radius [cm]
const double rWire = 25.e-4;
// Outer radius of the tube [cm]
const double rTube = 0.71;
// Voltages
const double vWire = 2730.;
const double vTube =    0.;
// Add the wire in the centre.
cmp.AddWire(0, 0, 2 * rWire, vWire, "s");
// Add the tube.
cmp.AddTube(rTube, vTube, 0, "t");
\end{lstlisting}
We want to calculate the signal induced on the wire. 
Using 
\begin{lstlisting}
cmp.AddReadout("s");
\end{lstlisting}
we tell \texttt{ComponentAnalyticField} to prepare the solution for the 
weighting field of the wire (which we have given the label ``s'' before).
 
Finally we assemble a \texttt{Sensor} object which acts as an 
interface to the transport classes discussed below.
\begin{lstlisting}
Sensor sensor;
// Calculate the electric field using the Component object cmp.
sensor.AddComponent(&cmp);
// Request signal calculation for the electrode named "s", 
// using the weighting field provided by the Component object cmp.
sensor.AddElectrode(&cmp, "s"); 
\end{lstlisting}
We further need to set the time interval within which the
signal is recorded and the granularity (bin width). 
In this example, we use use 1000 bins with a width of 0.5\,ns.
\begin{lstlisting}
const double tstep = 0.5;
const double tmin = -0.5 * tstep;
const unsigned int nbins = 1000;
sensor.SetTimeWindow(tmin, tstep, nbins);
\end{lstlisting}

\subsubsection{Drift lines from a track}
We use Heed (Sec.~\ref{Sec:Heed}) to simulate the ionisation 
produced by a charged particle crossing the tube 
(a 170\,GeV muon in our example).
\begin{lstlisting}
TrackHeed track;
track.SetParticle("muon");
track.SetEnergy(170.e9);
track.SetSensor(&sensor);
\end{lstlisting}
The drift lines of the electrons created along the track are simulated 
along the track are calculated using Runge-Kutta-Fehlberg (RKF) integration,
implemented in the class \texttt{DriftLineRKF}.
This method uses the previously computed tables of transport parameters to 
calculate drift lines and multiplication. 
\begin{lstlisting}
DriftLineRKF drift;
drift.SetSensor(&sensor);
// Switch on signal calculation.
drift.EnableSignalCalculation();
\end{lstlisting}
Let us consider a track that passes at a distance of 3\,mm
from the wire centre. After simulating the passage of the charged particle,
we loop over the ``clusters'' 
(\ie~the ionizing collisions of the primary particle)
along the track and calculate a drift line for each electron produced in 
the cluster.
\begin{lstlisting}
const double rTrack = 0.3;
const double x0 = rTrack;
const double y0 = -sqrt(rTube * rTube - rTrack * rTrack);
track.NewTrack(x0, y0, 0, 0, 0, 1, 0);
// Loop over the clusters along the track.
double xc = 0., yc = 0., zc = 0., tc = 0., ec = 0., extra = 0.;
int nc = 0;
while (track.GetCluster(xc, yc, zc, tc, nc, ec, extra)) {
  // Loop over the electrons in the cluster.
  for (int k = 0; k < nc; ++k) {
    double xe = 0., ye = 0., ze = 0., te = 0., ee = 0.;
    double dx = 0., dy = 0., dz = 0.;
    track.GetElectron(k, xe, ye, ze, te, ee, dx, dy, dz);
    drift.DriftElectron(xe, ye, ze, te);
  }
}
\end{lstlisting}
As a check whether the simulation is doing something sensible, 
it can be useful to visualize the drift lines. 
Before simulating the charged particle track and
the electron drift lines, we have to instruct \texttt{TrackHeed} and
\texttt{DriftLineRKF} to pass the coordinates of the clusters and the 
points along the drift line to a \texttt{ViewDrift} object
which then takes care of plotting them.
\begin{lstlisting}
// Create a canvas.
cD = new TCanvas("cD", "", 600, 600);
ViewDrift driftView;
driftView.SetCanvas(cD);
drift.EnablePlotting(&driftView);
track.EnablePlotting(&driftView);
\end{lstlisting}
We use the class \texttt{ViewCell} to draw the
outline of the tube and the position of the wire on the same plot as the
drift lines.
\begin{lstlisting}
ViewCell cellView;
cellView.SetCanvas(cD);
cellView.SetComponent(&cmp);
\end{lstlisting}
After we've simulated all drift lines from a charged particle 
track, we create a plot using
\begin{lstlisting}
cellView.Plot2d();
constexpr bool twod = true;
constexpr bool drawaxis = false;
driftView.Plot(twod, drawaxis);
\end{lstlisting}

Using the class \texttt{ViewSignal}, we plot the current induced on 
the wire by the drift lines simulated in the previous step.
\begin{lstlisting}
ViewSignal signalView;
signalView.SetSensor(&sensor);
signalView.PlotSignal("s");
\end{lstlisting}

\subsection{GEM}\label{Sec:ExampleGem}
In this example, we will simulate the drift of electrons and ions 
in a standard GEM \cite{Sauli2016}, 
which consists of a 50\,\textmu{m} thick kapton foil 
coated on both sides with a 5\,\textmu{m} layer of copper, 
with a hexagonal pattern of holes (outer hole diameter: 70\,\textmu{m}, 
inner hole diameter: 50\,\textmu{m}, pitch: 140\,\textmu{m}) 
etched into the foil.
\subsubsection{Field map}
As a first step, we need to calculate the electric field in the GEM. 
Currently this calculation has to be performed using an external field 
solver like Ansys \cite{ANSYS} or Elmer \cite{Elmer}. In this example, 
we use Ansys, but the steps for importing and using an Elmer field map 
are very similar.

In the following we assume that the output files resulting from the 
Ansys run are located in the current working directory.
The initialisation of \texttt{ComponentAnsys123} consists of 
\begin{itemize}
  \item
  loading the mesh (\texttt{ELIST.lis}, \texttt{NLIST.lis}), 
  the list of nodal solutions (\texttt{PRNSOL.lis}), and the 
  material properties (\texttt{MPLIST.lis});
  \item
  specifying the length unit of the values given in the 
  \texttt{.LIS} files;
  \item
  setting the appropriate periodicities/symmetries.
\end{itemize}
\begin{lstlisting}
ComponentAnsys123 fm;
// Load the field map.
fm.Initialise("ELIST.lis", "NLIST.lis", "MPLIST.lis", "PRNSOL.lis", "mm");
// Set the periodicities
fm.EnableMirrorPeriodicityX();
fm.EnableMirrorPeriodicityY();
// Print some information about the cell dimensions.
fm.PrintRange();
\end{lstlisting}
We need to apply mirror periodicity in $x$ and $y$
in \texttt{ComponentAnsys123} since we had exploited the symmetry 
of the geometry and modelled only half a hole in Ansys.

Using the class \texttt{ViewField}, we do a visual inspection of the 
field map to make sure it makes sense. 
We first make a plot of the potential along the hole axis ($z$ axis).
\begin{lstlisting}
ViewField fieldView;
fieldView.SetComponent(&fm);
// Plot the potential along the hole axis.
fieldView.PlotProfile(0., 0., 0.02, 0., 0., -0.02);
\end{lstlisting}
We also make a contour plot of the potential in the $x - z$ plane.
\begin{lstlisting}
const double pitch = 0.014;
// Set the normal vector (0, -1, 0) of the viewing plane.
fieldView.SetPlane(0., -1., 0., 0., 0., 0.);
fieldView.SetArea(-pitch / 2., -0.02, pitch / 2., 0.02);
fieldView.SetVoltageRange(-160., 160.);
fieldView.PlotContour();
\end{lstlisting}
Next we create a \texttt{Sensor} and add the field map 
component to it
\begin{lstlisting}
Sensor sensor;
sensor.AddComponent(&fm);
\end{lstlisting}
Normally, particles are transported until they exit the mesh. 
To speed up the calculation we restrict the drift region to 
$-100,\textmu\text{m} < z < +250\,\textmu\text{m}$.
\begin{lstlisting}
sensor.SetArea(-5 * pitch, -5 * pitch, -0.01, 5 * pitch,  5 * pitch,  0.025);
\end{lstlisting}
\subsubsection{Gas}
We use a gas mixture of 80\% argon and 20\% CO\(_{2}\).
\begin{lstlisting}
MediumMagboltz gas;
gas.SetComposition("ar", 80., "co2", 20.);
// Set temperature [K] and pressure [Torr].
gas.SetTemperature(293.15);
gas.SetPressure(760.);
\end{lstlisting}
\paragraph{Electron collision rates}
In this example, we will simulate electron avalanches using 
a ``microscopic'' Monte Carlo method,
based on the electron-atom/molecule cross-sections in the database of the 
\textsc{Magboltz program}. 
As discussed in more detail in Sec.~\ref{Sec:MicroscopicTracking},
the algorithm takes as input the collision rates (as function of the 
electron's kinetic energy) for each scattering mechanism 
that can take place in the gas.
The preparation of the tables of collision rates and the interpolation 
in these tables is done by the class \texttt{MediumMagboltz}, which -- as
the name suggests -- provides an interface to \textsc{Magboltz}.

In \texttt{MediumMagboltz} the collision rates are stored on an
evenly spaced energy grid. The max. energy can be set by the user.
For avalanche calculations, 50 -- 200\,eV is usually a reasonable choice.
\begin{lstlisting}
gas.SetMaxElectronEnergy(200.);
gas.Initialise();
\end{lstlisting}
\paragraph{Penning transfer}
Argon includes a number of excitation levels with an excitation energy
exceeding the ionisation potential of CO$_{2}$ (13.78\,eV).
These excited states can contribute to the gain,
since (part of) the excitation energy can be transferred
to a CO$_2$ molecule through collisions or by photo-ionisation.

In the simulation, this so-called Penning effect can be described 
in terms of a probability $r$ that an excitation is converted to an 
ionising collision (Sec.~\ref{Sec:PenningTransfer}).
\begin{lstlisting}
// Penning transfer probability.
const double rPenning = 0.57;
// Mean distance from the point of excitation.
const double lambdaPenning = 0.;
gas.EnablePenningTransfer(rPenning, lambdaPenning, "ar");
\end{lstlisting}
\paragraph{Ion transport properties}
Unlike electrons, ions cannnot be tracked microscopically in Garfield++.
Moreover, there is no program like \textsc{Magboltz} that can compute  
the macroscopic transport properties (such as the drift velocity),
so we have to provide these data ourselves.
In this example, we use the mobility of Ar$^{+}$ ions in Ar
as an approximation because there is no literature data for drift in the
mixture.
Example files with mobility data for various pure gases are located
in the \texttt{Data} directory of the project.
Note that the \texttt{LoadIonMobility} method does not prefix
the file name with a directory.
If your mobility file is not located in your current working directory,
then you have to specify a fully qualified file name.
\begin{lstlisting}
const std::string path = getenv("GARFIELD\_HOME");
gas.LoadIonMobility(path + "/Data/IonMobility\_Ar+\_Ar.txt");
\end{lstlisting}
\paragraph{Associating the gas to a field map region}
In order to track a particle through the detector we have to 
tell \texttt{ComponentAnsys123} which field map material corresponds 
to which \texttt{Medium}.
The gas can be distinguished from the other materials
(here: kapton and copper) by its dielectric constant, in our case
$\varepsilon = 1$.
\begin{lstlisting}
const unsigned int nMaterials = fm->GetNumberOfMaterials();
for (unsigned int i = 0; i < nMaterials; ++i) {
  const double eps = fm.GetPermittivity(i);
  if (fabs(eps - 1.) < 1.e-3) fm.SetMedium(i, gas);
}
// Print a list of the field map materials (for information).
fm.PrintMaterials();
\end{lstlisting}

\subsubsection{Electron transport}
Microscopic tracking is handled by the class 
\texttt{AvalancheMicroscopic} (Sec.~\ref{Sec:MicroscopicTracking}).
\begin{lstlisting}
AvalancheMicroscopic aval;
aval.SetSensor(&sensor);
\end{lstlisting}
We are now ready to simulate an electron avalanche in the GEM.
We place the initial electron 200\,\textmu{m} above centre of the GEM hole 
and set its initial energy to a typical energy in the electric field 
of the drift gap.
\begin{lstlisting}
// Initial position [cm] and starting time [ns]
double x0 = 0., y0 = 0., z0 = 0.02;
double t0 = 0.;
// Initial energy [eV]
double e0 = 0.1;
// Initial direction 
// In case of a null vector, the initial direction is randomized.
double dx0 = 0., dy0 = 0., dz0 = 0.;
// Calculate an electron avalanche.
aval.AvalancheElectron(x0, y0, 0, t0, e0, dx0, dy0, dz0);
\end{lstlisting}
After the calculation, we can extract information such as
the number of electrons/ions produced in the avalanche
and the start- and endpoints of all electron trajectories. 
\begin{lstlisting}
int ne, ni;
// Get the number of electrons and ions in the avalanche.
aval.GetAvalancheSize(ne, ni);
// Get the number of electron drift lines.
int np = aval->GetNumberOfElectronEndpoints();
// Initial position and time
double x1, y1, z1, t1;
// Final position and time
double x2, y2, z2, t2;
// Initial and final energy
double e1, e2;
// Flag indicating why the tracking of an electron was stopped.
int status;
// Loop over the avalanche electrons.
for (int i = 0; i < np; ++i) {
  aval.GetElectronEndpoint(i, x1, y1, z1, t1, e1,
                              x2, y2, z2, t2, e2, status);
} 
\end{lstlisting}
\paragraph{Ion transport}
For tracking the ions, we use the class \texttt{AvalancheMC}, 
which takes as input the macroscopic drift velocity as 
function of the electric field and simulates drift lines using a 
Monte Carlo technique (Sec.~\ref{Sec:DriftLineMC}).
\begin{lstlisting}
AvalancheMC drift;
drift.SetSensor(sensor);
drift.SetDistanceSteps(2.e-4);
\end{lstlisting}
After simulating an electron avalanche, we loop over all the electron
trajectories, and calculate an ion drift line starting from the same
initial point as the electron.
\begin{lstlisting}
// Loop over the avalanche electrons.
for (int i = 0; i < np; ++i) {
  aval.GetElectronEndpoint(i, x1, y1, z1, t1, e1,
                              x2, y2, z2, t2, e2, status);
  drift.DriftIon(x1, y1, z1, t1);
\end{lstlisting}
We can subsequently retrieve the endpoint of the ion drift line using
\begin{lstlisting}
double xi1, yi1, zi1, ti1;
double xi2, yi2, zi2, ti2;
drift.GetIonEndpoint(0, xi1, yi1, zi1, ti1,
                        xi2, yi2, zi2, ti2, status);
\end{lstlisting}
\subsubsection{Visualizing the drift lines}
To plot the electron and ion drift lines together with the geometry, 
we use the classes \texttt{ViewDrift} and \texttt{ViewFEMesh}. 
When setting up the \texttt{AvalancheMicroscopic} and \texttt{AvalancheMC} 
objects, we need to switch on the storage of the drift line points and 
attach a pointer to a \texttt{ViewDrift} object.
\begin{lstlisting}
ViewDrift driftView;
aval.EnablePlotting(&driftView);
drift.EnablePlotting(&driftView);
\end{lstlisting}
After having calculated the electron avalanche and ion drift lines, we create a plot using the snippet of code below.
\begin{lstlisting}
ViewFEMesh* meshView = new ViewFEMesh();
meshView->SetArea(-2 * pitch, -2 * pitch, -0.02,
                   2 * pitch,  2 * pitch,  0.02);
meshView->SetComponent(&fm);
// x-z projection
meshView->SetPlane(0, -1, 0, 0, 0, 0);
meshView->SetFillMesh(true); 
// Set the color of the kapton.
meshView->SetColor(2, kYellow + 3);
meshView->EnableAxes();
meshView->SetViewDrift(&driftView);
meshView->Plot();
\end{lstlisting}
\subsection{Silicon sensor}\label{Sec:ExampleSilicon}
In this example, we will simulate the signal in a 100\,\textmu{m} thick 
planar silicon sensor due to the passage of a charged particle. 
We will adopt a coordinate system where the back side of the sensor is at 
$y = 0$ and the front side (\ie the strip or pixel side) is at $y = 100$\,\textmu{m}.

\subsubsection{Transport properties}
We start by creating a \texttt{MediumSilicon} object, 
which provides the transport parameters (\eg drift velocity and 
diffusion coefficients) of electrons and holes as function 
of the electric field (and, in general, also the magnetic field, 
but we will assume that it is zero in this example).
\begin{lstlisting}
MediumSilicon si;
// Set the temperature [K].
si.SetTemperature(293.15);
\end{lstlisting}
Unless the user overrides the default behaviour 
(by providing a table of velocities at different electric fields), 
\texttt{MediumSilicon} calculates the drift velocities according to 
analytic parameterizations. 
A description of the mobility models is given in Sec.~\ref{Sec:Silicon}.
In this example, we will use the default parameterizations, 
which correspond to the default models in Sentaurus Device \cite{Synopsys}. 
The diffusion coefficients are calculated according to the Einstein relation.

\subsubsection{Geometry}
As a next step, we define the active volume, 
which in our case is simply a box with a length of $d = 100$\,\textmu{m} 
along $y$, centred at $y = 50$\,\textmu{m}, and made of silicon. 
To describe the shape of our detector, 
we therefore create a \texttt{SolidBox} object.
\begin{lstlisting}
// Thickness of the silicon [cm]
constexpr double d = 100.e-4;
SolidBox box(0, 0.5 * d, 0, 2 * d, 0.5 * d, 2 * d);
\end{lstlisting}
We then create a \texttt{GeometrySimple} object, and attach the box to it 
(\ie we pass it a pointer to the \texttt{SolidBox} object), 
together with the medium inside 
(\ie a pointer to the \texttt{MediumSilicon} object).
\begin{lstlisting}
// Set up the geometry.
GeometrySimple geo;
geo.AddSolid(&box, &si);
\end{lstlisting}

\subsubsection{Electric field}
For accurate calculations of the electric field in a segmented silicon sensor, 
one normally uses TCAD device simulation programs such as 
Synopsys Sentaurus Device \cite{Synopsys}. 
In the present example, we will follow a simplified approach 
and approximate the electric field by that of an overdepleted pad sensor. 
In that case, the $x$ and $z$ components of the electric field vanish, 
and the $y$ component varies linearly between
\begin{equation*}
E_y = \frac{V_{\text{bias}} - V_{\text{dep}}}{d}
\end{equation*}
at the back side of the sensor ($y = 0$) and
\begin{equation*}
E_y = \frac{V_{\text{bias}} + V_{\text{dep}}}{d}
\end{equation*}
at the front side of the sensor ($y = d$), where 
$V_{\text{dep}}$ is the depletion voltage of the sensor and 
$V_{\text{bias}}$ is the applied bias voltage. 
In this example, we will use $V_{\text{dep}} = -20$\,V and 
$V_{\text{bias}} = -50$\,V.

In order to use this expression for the electric field in our simulation, 
we need to write a small function
\begin{lstlisting}
void eLinear(const double /*x*/, const double /*y*/, const double /*z*/,
             double& ex, double& ey, double& ez) {

  // Bias voltage [V]
  constexpr double vbias = -50.;
  // Depletion voltage [V]
  constexpr double vdep = -20.;
  ex = ez = 0.;
  ey = (vbias - vdep) / d + 2 * y * vdep / (d * d);
}
\end{lstlisting}
and set up a \texttt{ComponentUser} object 
which delegates the calculation of the electric field to this function.
\begin{lstlisting}
ComponentUser linearField;
linearField.SetGeometry(&geo);
linearField.SetElectricField(eLinear);
\end{lstlisting}
A pointer to this \texttt{Component} is then passed to a \texttt{Sensor}
which acts as an interface to the transport classes.
\begin{lstlisting}
Sensor sensor;
sensor.AddComponent(&linearField);
\end{lstlisting}

\subsubsection{Weighting field}

For signal simulations, 
we need to know not only the actual electric field in the sensor, 
but also the weighting field of the electrode for which we want to 
calculate the induced current (Chapter~\ref{Chap:Signals}).

In this example, 
we will use an analytic expression for the weighting field of a strip, 
as implemented in the class \texttt{ComponentAnalyticField}. 
We thus create a \texttt{ComponentAnalyticField} object, 
define the equipotential planes ($y = 0$ and $y = d$) 
and set the voltages at these planes to ground and $V = V_{\text{bias}}$. 
We will not use this class to calculate the ``real'' electric field 
in the sensor though, so the voltage settings don't actually matter 
for our purposes.
\begin{lstlisting}
ComponentAnalyticField wField;
wField.SetGeometry(&geo);
wField.AddPlaneY(0, vbias, "back");
wField.AddPlaneY(d, 0, "front");
\end{lstlisting}
We now define a strip (55\,\textmu{m} width, centred at $x = 0$) 
on the front side of the sensor and request the calculation of its 
weighting field (which we label ``strip'').
\begin{lstlisting}
constexpr double pitch = 55.e-4;
constexpr double halfpitch = 0.5 * pitch;
wField.AddStripOnPlaneY('z', d, -halfpitch, halfpitch, "strip");
wField.AddReadout("strip");
\end{lstlisting}
Similarly we could have set up the weighting field of a pixel electrode.
\begin{lstlisting}
wField.AddPixelOnPlaneY(d, -halfpitch, halfpitch, -halfpitch, halfpitch, "pixel");
wField.AddReadout("pixel");
\end{lstlisting}

Finally, we need to instruct the \texttt{Sensor} 
to use the strip weighting field we just prepared 
for computing the induced signal
\begin{lstlisting}
// Request signal calculation for the electrode named "strip",
// using the weighting field provided by the Component object wField.
sensor.AddElectrode(&wField, "strip");
\end{lstlisting}
and we need to set the granularity with which we want to record the signal 
(in our example: 1000 bins between 0 and 10\,ns).

\begin{lstlisting}
// Set the time bins.
const unsigned int nTimeBins = 1000;
const double tmin =  0.;
const double tmax = 10.;
const double tstep = (tmax - tmin) / nTimeBins;
sensor.SetTimeWindow(tmin, tstep, nTimeBins);
\end{lstlisting}

\subsubsection{Primary ionization and charge carrier transport}

We use Heed (Sec.~\ref{Sec:Heed}) to simulate the electron/hole pairs 
produced by a 180\,GeV\,/\,$c$ charged pion traversing the sensor.
\begin{lstlisting}
TrackHeed track;
track.SetSensor(&sensor);
// Set the particle type and momentum [eV/c].
track.SetParticle("pion");
track.SetMomentum(180.e9);
\end{lstlisting}
For transporting the electrons and holes, we use the class \texttt{AvalancheMC}. 
When setting up the \texttt{AvalancheMC} object, 
we need to set the step size used for the drift line calculation 
to a reasonable value. In this example, we use steps of 1\,\textmu{m}. 
This means that at each step, the electron/hole will be propagated by 
1\,\textmu{m} in the direction of the drift velocity at the local field, 
followed by a random step based on the diffusion coefficient.
\begin{lstlisting}
AvalancheMC drift;
drift.SetSensor(&sensor);
// Set the step size [cm].
drift.SetDistanceSteps(1.e-4);
drift.EnableSignalCalculation();
\end{lstlisting}
We are now ready to run the simulation. In the snippet below, 
we simulate a perpendicularly incident charged particle track 
passing through the centre of the strip ($x = 0$), 
loop over the electron/hole pairs produced by the particle, 
and simulate a drift line for each electron and hole.
\begin{lstlisting}
double x0 = 0., y0 = 0., z0 = 0., t0 = 0.;
double dx = 0., dy = 1., dz = 0.; 
track.NewTrack(x0, y0, z0, t0, dx, dy, dz);
double xc = 0., yc = 0., zc = 0., tc = 0., ec = 0., extra = 0.;
int ne = 0;
// Retrieve the clusters along the track.
while (track.GetCluster(xc, yc, zc, tc, ne, ec, extra)) {
  // Loop over the electrons in the cluster.
  for (int j = 0; j < ne; ++j) {
    double xe = 0., ye = 0., ze = 0., te = 0., ee = 0.;
    double dxe = 0., dye = 0., dze = 0.;
    track.GetElectron(j, xe, ye, ze, te, ee, dxe, dye, dze);
    // Simulate the electron and hole drift lines.
    drift.DriftElectron(xe, ye, ze, te);
    drift.DriftHole(xe, ye, ze, te);
  }
}
\end{lstlisting}
To check whether the results are sensible, 
it can be instructive to visualize the drift lines 
using the class \texttt{ViewDrift}.
\begin{lstlisting}
ViewDrift driftView;
driftView.SetArea(-0.5 * d, 0, -0.5 * d, 0.5 * d, d, 0.5 * d);
track.EnablePlotting(&driftView);
drift.EnablePlotting(&driftView);
\end{lstlisting}
With the plotting option switched on, \texttt{AvalancheMC} 
will pass the coordinates of all drift line points 
to a \texttt{ViewDrift} object. 
After having simulated all drift lines from a track, we can create a plot using
\begin{lstlisting}
constexpr bool twod = true;
driftView.Plot(twod);
\end{lstlisting}
Plotting the drift lines can slow down the execution time quite a bit, so it is advisable to switch it off when simulating a large number of tracks.

\subsubsection{Retrieving the signal}
After having simulated the charge carrier drift lines, 
we can plot the induced current using the class \texttt{ViewSignal}.
\begin{lstlisting}
ViewSignal signalView;
signalView.SetSensor(&sensor);
constexpr bool plotTotalSignal = true;
constexpr bool plotElectronSignal = false;
constexpr bool plotHoleSignal = false;
signalView.PlotSignal("strip", plotTotalSignal, plotElectronSignal, plotHoleSignal);
\end{lstlisting}

To post-process the induced current pulse, one can convolute 
it with a transfer function that describes the response 
of the front-end electronics.

Often it can also be useful to save the signal to a file. 
An example for doing so is given in the code snippet below.
\begin{lstlisting}
std::ofstream outfile;
outfile.open("signal.txt", std::ios::out);
for (unsigned int i = 0; i < nTimeBins; ++i) {
  const double t = (i + 0.5) * tstep;
  const double f = sensor.GetSignal(label, i);
  const double fe = sensor.GetElectronSignal(label, i);
  const double fh = sensor.GetIonSignal(label, i);
  outfile << t << "  " << f << "  " << fe << "  " << fh << "\n";
}
\end{lstlisting}

