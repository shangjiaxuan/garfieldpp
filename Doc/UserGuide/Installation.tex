\section{Prerequisites}

To build Garfield++ and a project that depends on it you need to have the following software correctly installed and configured on the user machine:

\begin{itemize}
	\item ROOT 6, possibly at least ROOT 6.20. For installation instructions 
	see \url{https://root.cern.ch/building-root} or
	\url{https://root.cern.ch/downloading-root}. 
	Declare the \texttt{ROOTSYS} environment variable to point to the base folder of the ROOT installation. 
	This operation is typically performed by the ROOT initialization script, called \texttt{thisroot.sh} therefore executing that script will also correctly initialize the right environment variables to install Garfield
	\item GSL\footnote{https://www.gnu.org/software/gsl/} (GNU Scientific Library)
	\item CMake 3.9\footnote{https://cmake.org/} or later version 
	\item A C++ compiler compatible with the same C++ standard with which ROOT was compiled.
	\item A Fortran compiler
	\item (Optional) OpenMP\footnote{https://openmp.org} to enable some additional parallel computations 
\end{itemize}

\section{Downloading the source code}

The source code is managed by a git repository hosted on the CERN GitLab\footnote{https://gitlab.cern.ch/help/gitlab-basics/start-using-git.md} server,
\url{https://gitlab.cern.ch/garfield/garfieldpp}.

Choose a folder where the source code is to be downloaded. Note that the chosen folder must be empty or non-existing. We will identify this folder with an environment variable named \texttt{<GARFIELD\_HOME>}. Note that this is not strictly required and you can simply replace the chosen path in all the following commands where that variable appears. To define that variable in the bash shell family

\begin{lstlisting}[language=bash]
	export GARFIELD_HOME=/home/git/garfield
\end{lstlisting} 
(replace \texttt{/home/git/garfield} by the path of your choice).

For (t)csh-type shells, type
\begin{lstlisting}[language=csh]
	setenv GARFIELD_HOME /home/git/garfield
\end{lstlisting}

Download the code from the repository, either 
using SSH access\footnote{See \url{https://gitlab.cern.ch/help/gitlab-basics/create-your-ssh-keys.md}
	for instructions how to create and upload the SSH keys for gitlab.} 
\begin{lstlisting}[language=bash]
git clone ssh://git@gitlab.cern.ch:7999/garfield/garfieldpp.git $GARFIELD_HOME
\end{lstlisting}
or HTTPS access
\begin{lstlisting}[language=bash]
git clone https://gitlab.cern.ch/garfield/garfieldpp.git $GARFIELD_HOME
\end{lstlisting}

To update the source code with the latest changes, from the \texttt{GARFIELD\_HOME} folder you can run the following command:
\begin{lstlisting}[language=bash]
	git pull
\end{lstlisting}

\section{Building the project}

Garfield++ uses the CMake build generator to create the actual build system that it is used to compile the binaries of the programs we want to build. CMake is a cross-platform scripting language that allows to define the rules to build and install a complex project and takes care to generate the low-level files necessary to the build system of choice. On Linux systems that build system typically defaults to the \texttt{GNU Make} program. We will therefore assume that\texttt{GNU Make} is used to build Garfield++.

The process is divided in two phases. During the build phase, the source files will be used to generate the necessary binaries, along with a large quantities of temporary files, necessary in the intermediate phases of the build. In the install phase only the necessary files are copied to the final destination.

First we will create a build directory, and move into it to execute the first phase. This can be any folder on the user filesystem. For simplicity we will use a subfolder of the place where the source code was downloaded

\begin{lstlisting}[language=bash]
mkdir $GARFIELD_HOME/build
cd build
\end{lstlisting}

Inside the build directory we can run CMake to generate the build system.

\begin{lstlisting}[language=bash]
cmake $GARFIELD_HOME
\end{lstlisting}

The build can be customized in several ways through CMake by defining a set of internal variables that modify the output accordingly. To set a new value for a CMake variable one can use the \texttt{-D<var>=<value>} syntax at the command line. The most relevant parameters that a user may want to customize are:

\begin{itemize}
	\item The installation folder. By default Garfield is installed in the folder pointed to by the environment variable \texttt{GARFIELD\_INSTALL} or, if that variable is missing, in a subfolder of the source directory, that is \texttt{\$GARFIELD\_HOME/install}. To install it elsewhere define the \texttt{CMAKE\_INSTALL\_PREFIX} variable, using a similar command with an appropriate target folder:
\begin{lstlisting}[language=bash]
cmake -DCMAKE_INSTALL_PREFIX=/home/mygarfield $GARFIELD_HOME
\end{lstlisting} 
	\item The debug and optimization mode. This is controlled by the CMake variable \texttt{CMAKE\_BUILD\_TYPE} which can have one of the following values: 
	\begin{description}
		\item[Release] Enables all the compiler optimizations to achieve the best performance
		\item[Debug] Disables all the compiler optimizations and add the debug symbols to the binary to be able to use an external debugger on the code
		\item[RelWithDebInfo] Enables all the compiler optimization but stores the debug symbols in the final binaries. This increases the binary size and may reduce the performances
		\item[MinSizeRel] Enables all the compiler optimization necessary to obtain the smaller executable possible
	\end{description}
	By default Garfield++ is build in \emph{Release} mode. To add the debugging symbol use the following command before building
\begin{lstlisting}[language=bash]
cmake -DCMAKE_BUILD_TYPE=RelWithDebInfo $GARFIELD_HOME
\end{lstlisting}
\end{itemize}

Those variables and many other can be set through a textual or graphical user interface, \texttt{ccmake} or \texttt{cmake-gui} respectively, that need to be installed separately.

Once CMake generated the build system, you can execute the following to compile and install Garfield++

\begin{lstlisting}[language=bash]
make && make install
\end{lstlisting}

Once the installation is done, Garfield requires the definition of an environment variable named \texttt{HEED\_DATABASE} to identify the location of the Heed cross-section database, located in \texttt{share/Heed/database} subfolder of the installation path. Additionally, to build applications that make use of Garfield it may be convenient to append the installation path to an environment variable named \texttt{CMAKE\_PREFIX\_PATH}. To simplify all this the build procedure generates a shell script, named \texttt{setupGarfield.sh}, located in the \texttt{share/Garfield} subfolder of the installation path, that correctly define all those variables. You can append the execution of that script to your shell initialization script, e.g. \texttt{.bashrc} for the Bash shell to setup Garfield automatically.

After updating the source code you can run the \texttt{make} command from the build folder, to update the build. Sometimes it may be necessary to restart from a clean slate, in which case one can remove the build folder completely and restart the procedures of this section.

\section{Building an application}
The recommended way to build a Garfield++-based application is 
using CMake. Let us consider as an example the program \texttt{gem.C} 
(see Sec.~\ref{Sec:ExampleGem}) 
which together with the corresponding \texttt{CMakeLists.txt} can be 
found in the directory \texttt{Examples/Gem} of the source tree. 
As a starting point, we assume that you have built Garfield++ using the 
instructions above and set up the necessary environment variables. 

\begin{itemize}
  \item
  To keep the source tree clean, and since you will probably want to modify 
  the program according to your needs, it is a good idea to copy 
  the folder to another location.
\begin{lstlisting}[language=bash]
cp -r $GARFIELD_HOME/Examples/Gem .
\end{lstlisting} 
  \item
  Create a build directory.
\begin{lstlisting}[language=bash]
mkdir Gem/build; cd Gem/build
\end{lstlisting} 
  \item
  Run CMake
\begin{lstlisting}[language=bash]
cmake ..
\end{lstlisting} 
  followed by
\begin{lstlisting}[language=bash]
make
\end{lstlisting}
  \item
  In addition to the executable (\texttt{gem}), the \texttt{build} 
  folder should now also contain the field map (\texttt{*.lis}) 
  files which have been copied there during the CMake step.
  \item
  To run the application, type
\begin{lstlisting}
./gem
\end{lstlisting}
\end{itemize}
%\subsection{GarfRoot}
%
%The Garfield++ classes can be used interactively within the ROOT shell.
%

