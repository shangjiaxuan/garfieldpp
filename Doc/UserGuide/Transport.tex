On a phenomenological level, 
the drift of charge carriers under the influence of an 
electric field \(\mathbf{E}\) and a magnetic field \(\mathbf{B}\) 
is described by the first order equation of motion
\begin{equation}\label{Eqn:FirstOrderEquationOfMotion}
  \dot{\mathbf{r}} = 
  \mathbf{v}_{\text{d}}\bigl(\mathbf{E}\left(\mathbf{r}\right), 
                             \mathbf{B}\left(\mathbf{r}\right)\bigr),
\end{equation}
where \(\mathbf{v}_{\text{d}}\) is the drift velocity. 
For the solution of \eqref{Eqn:FirstOrderEquationOfMotion}, 
two methods are available in Garfield++:
\begin{itemize}
  \item
  Runge-Kutta-Fehlberg integration, and
  \item
  Monte Carlo integration (\texttt{AvalancheMC}).
\end{itemize}

For accurate simulations of electron trajectories 
in small-scale structures 
(with characteristic dimensions comparable to the electron mean free path),
and also for detailed calculations of ionisation and excitation processes, 
transporting electrons on a microscopic level -- 
\textit{i.\,e.} based on the second-order equation of motion --
is the method of choice. 
Microscopic tracking of electrons is dealt with by the class 
\texttt{AvalancheMicroscopic}.  
\section{Runge-Kutta-Fehlberg integration}
\begin{table}
\centering
\caption{Coefficients of the Runge-Kutta-Fehlberg formula \cite{StoerBulirsch}.}
\label{Tab:CoefficientsRKF}
\begin{tabular}{c c c c} 
\toprule
                   & \multicolumn{3}{c}{$\beta_{k,i}$} \\
\diagbox{$k$}{$i$} & 0 & 1 & 2 \\
\midrule
1 & $\frac{1}{4}$ \\
2 & $\frac{-189}{800}$ & $\frac{729}{800}$ \\
3 & $\frac{214}{891}$ & $\frac{1}{33}$ & $\frac{650}{891}$ \\
\bottomrule
\end{tabular}
\quad
\begin{tabular}{c c c}
\toprule
$k$ & $C_{I,k}$ & $C_{II,k}$ \\
\midrule
0 & $\frac{214}{891}$ & $\frac{533}{2106}$ \\  
1 & $\frac{1}{33}$    & \\
2 & $\frac{650}{891}$ & $\frac{800}{1053}$ \\ 
3 &                   & $\frac{-1}{78}$ \\ 
\bottomrule
\end{tabular}
\end{table}
This method, implemented in the class \texttt{DriftLineRKF}, 
calculates a drift line by iterating over the following steps.
\begin{enumerate}
  \item
  Given a starting point $\mathbf{x}_{0}$, the velocity at the starting point, 
  and a time step $\Delta{t}$, 
  compute two estimates of the step to the next point on the drift line, 
  \begin{itemize}
    \item
      $\boldsymbol{\Delta{v}}_{\text{I}} = \sum\limits_{k=0}^{2}C_{\text{I}, k}\mathbf{v}_{d}\left(\mathbf{x}_{k}\right)$, accurate to second order, and
    \item
      $\boldsymbol{\Delta{v}}_{\text{II}} = \sum\limits_{k=0}^{3}C_{\text{II}, k}\mathbf{v}_{d}\left(\mathbf{x}_{k}\right)$, accurate to third order.
  \end{itemize}
  These two estimates are based on the drift velocity at the starting point 
  and the velocity at three new locations
  \begin{equation*} 
  \mathbf{x}_{k} = \mathbf{x}_{0} + \Delta{t}\sum\limits_{i = 0}^{k - 1}\beta_{k,i}\mathbf{v}_{d}\left(\mathbf{x}_{i}\right).
  \end{equation*}
  The values of the coefficients are shown in Table~\ref{Tab:CoefficientsRKF}. 
  \item
  The time step is updated by comparing the second and third order estimates 
  with the requested accuracy
  \begin{equation*}
    \Delta{t}' = \sqrt{\frac{\varepsilon\Delta{t}}{\left|
      \boldsymbol{\Delta{v}}_{\text{I}} - \boldsymbol{\Delta{v}}_{\text{II}}\right|}}.
  \end{equation*}
  \item
  The step is repeated if
  \begin{itemize}
    \item
    the time step shrinks by more than a factor five,
    \item
    the step size exceeds the maximum step length allowed (if such a limit is set),
  \end{itemize}
  \item
  The position is updated with the second order estimate.
  \item
  The velocity is updated according to the end-point velocity of the step, 
  which is one of the three velocity vectors that were computed under 1.
\end{enumerate}
The initial time step is estimated using 
$\Delta{t} = \varepsilon / \left|\mathbf{v}_{d}\left(\mathbf{x}_{0}\right)\right|$.
The parameter $\varepsilon$ used in this initial estimate and in step 2 of the above 
algorithm, can be set using
\begin{lstlisting}
void SetIntegrationAccuracy(const double eps);
\end{lstlisting} 

When traversing a large area with a very smooth field, the step size becomes large. 
If this is not desired, for instance because there is a fine structure behind the 
smooth area, then one should limit the step size using
\begin{lstlisting}
void SetMaximumStepSize(const double ms);
\end{lstlisting}
The maximum step size is recommended to be of order $1/10 - 1/20$ 
of the distance to be traversed. The default behaviour (no limit on the step size) 
can be reinstated using
\begin{lstlisting}
void UnsetMaximumStepSize();
\end{lstlisting}

By default, the drift line calculation is aborted if the drift line makes a bend 
sharper than 90$^{\circ}$. Such bends rarely occur in smooth fields, 
the most common case is a drift line that tries to cross a saddle point. 
This check can be switched on or off using
\begin{lstlisting}
void RejectKinks(const bool on = true);
\end{lstlisting} 

During the evaluation of the velocities $\mathbf{v}_{d}\left(\mathbf{x}_{k}\right)$
for the next step, checks are made to ensure that none of the probe points 
$\mathbf{x}_{k}$ are outside the active area and that no wire was crossed.
If a wire has been crossed during the step, another algorithm for stepping towards 
a wire takes over. The stepping towards the wire also starts when 
the distance between the particle position and a wire is less than
$n$ times the wire radius. The factor $n$ 
(``trap radius'') can be set in \texttt{ComponentAnalyticField} when defining the wire. 
If one of the points along the step is outside the active area,
the drift line is terminated by doing a last linear step towards the boundary.

Drift line calculations are started using 
\begin{lstlisting}
bool DriftElectron(const double x0, const double y0, const double z0, const double t0);
bool DriftHole(const double x0, const double y0, const double z0, const double t0);
bool DriftIon(const double x0, const double y0, const double z0, const double t0);
\end{lstlisting}
\begin{description}
  \item[x0, y0, z0, t0] initial position and time
\end{description}

The function 
\begin{lstlisting}
bool DriftPositron(const double x0, const double y0, const double z0, const double t0);
\end{lstlisting}
computes an electron drift line, but assuming that the electron has positive charge 
(which can be useful for determining isochrons). Analogously, the function
\begin{lstlisting}
bool DriftNegativeIon(const double x0, const double y0, const double z0, const double t0);
\end{lstlisting}
computes an ion drift line, assuming that the ion has negative charge.
% Note that the drift velocity vector is not anti-symmetric under charge inversion 
% if there is a non-zero E-perpendicular component of the B field.

After calculating a drift line, the multiplication and loss factors 
\begin{equation*}
  \exp\left(\int\alpha\text{d}s\right), \exp\left(-\int\eta\text{d}s\right)
\end{equation*}
along the drift line can be obtained using 
\begin{lstlisting}
void GetGain(const double eps = 1.e-4);
void GetLoss(const double eps = 1.e-4);
\end{lstlisting}
\begin{description}
  \item[eps] parameter determining the accuracy of the integration.
\end{description}
Both functions use an adaptive Simpson-style integration.

Similarly, the function
\begin{lstlisting}
void GetArrivalTimeSpread(const double eps = 1.e-4);
\end{lstlisting}
returns the $\sigma$ of the arrival time distribution calculated by 
integrating the longitudinal diffusion coefficient along the drift line. 

The points along the most recent drift line are accessible through the functions
\begin{lstlisting}
unsigned int GetNumberOfDriftLinePoints() const;
void GetDriftLinePoint(const unsigned int i, 
                       double& x, double& y, double& z, double& t) const;
\end{lstlisting}
\begin{description}
  \item[i] index of the point.
  \item[x, y, z, t] coordinates and time of the point. 
\end{description}
If one is simply interested in the end point and status flag of the 
current drift line, the function
\begin{lstlisting}
void GetEndPoint(double& x, double& y, double& z, double& t, int& st) const;
\end{lstlisting}
can be used. A list of the status codes 
is given in Table~\ref{Tab:DriftLineStatusCodes}.

By default, the induced current is calculated for each drift line.
This can be activated or deactivated using
\begin{lstlisting}
void EnableSignalCalculation(const bool on = true);
\end{lstlisting}

For electron drift lines, multiplication is taken into account in the signal 
calculation. For this purpose, after calculating a drift line
the number of electrons and ions at each point of the line is calculated 
by integrating the Townsend and attachment coefficient along the line. 
For a given starting point, 
the number of electrons at the end of the drift line is thus given by 
\begin{equation*}
  n_{e} = \exp\left(\int\left(\alpha - \eta\right)\text{d}s\right).
\end{equation*}
The multiplication factor can be set explicitly using
\begin{lstlisting}
void SetGainFluctuationsFixed(const double gain = -1.);
\end{lstlisting}
\begin{description}
  \item[gain] multiplication factor to be used. 
  If the provided value is $< 1$, the multiplication factor 
  obtained by integrating $\alpha - \eta$ along the 
  drift line is used instead (as is the default).
\end{description} 

In order to take fluctuations of the avalanche size into account in 
the signal calculation, the number of electrons in the avalanche 
can be sampled from a P\'olya distribution \cite{Alkhazov1970}
\begin{equation*}
 \overline{n}P_{n} = \frac{\left(\theta + 1\right)^{\theta + 1}}{\Gamma\left(\theta + 1\right)}\left(\frac{n}{\overline{n}}\right)^{\theta}\text{e}^{-\left(\theta + 1\right)n/\overline{n}},
\end{equation*}
where $P_{n}$ is the probability that the avalanche comprises $n$ electrons, 
$\overline{n}$ is the mean avalanche size, and $\theta$ is a parameter 
controlling the shape of the distribution. For $\theta = 0$ one obtains an 
exponential distribution, while with increasing $\theta$ the distribution becomes 
more and more ``rounded''. The simulation of gain fluctuations can be enabled using 
the function 
\begin{lstlisting}
void SetGainFluctuationsPolya(const double theta, const double mean = -1.);
\end{lstlisting}
\begin{description}
  \item[theta] shape parameter $\theta$ of the P\'olya distribution.
  \item[mean] mean avalanche size $\overline{n}$. 
  If the provided value is $< 1$, the multiplication factor 
  obtained by integrating $\alpha - \eta$ along the 
  drift line is used instead (as is the default).
\end{description}

\section{Monte Carlo integration}\label{Sec:DriftLineMC}
In the class \texttt{AvalancheMC}, Eq.~\eqref{Eqn:FirstOrderEquationOfMotion}
is integrated in a stochastic manner:
\begin{itemize}
  \item
  a step of length \(\Delta{s} = v_{\text{d}}\Delta{t}\) 
  in the direction of the
  drift velocity \(\mathbf{v}_{\text{d}}\) 
  at the local field is calculated (with either the 
  time step \(\Delta{t}\) or the distance \(\Delta{s}\) 
  being specified by the user);
  \item
   a random diffusion step
   is sampled from three uncorrelated Gaussian distributions
   with standard deviation \(\sigma_{L} = D_{L}\sqrt{\Delta{s}}\)
   for the component parallel to the drift velocity and
   standard deviation
   \(\sigma_{T} = D_{T}\sqrt{\Delta{s}}\) for the two
   transverse components;
   \item
   the two steps are added vectorially and the location is updated.
\end{itemize}
The functions for setting the step size are 
\begin{lstlisting}
void SetTimeSteps(const double d = 0.02);
void SetDistanceSteps(const double d = 0.001);
void SetCollisionSteps(const int n = 100);
\end{lstlisting} 
In the first case the integration is done 
using fixed time steps (default: 20\,ps), 
in the second case using fixed distance steps (default: 10\,\textmu{m}). 
Calling the third function instructs the class to 
do the integration with exponentially distributed time steps 
with a mean equal to a multiple of the ``collision time'' 
\begin{equation*}
  \tau = \frac{m v_{d}}{q E}.
\end{equation*}
The third method is activated by default.

Instead of making simple straight-line steps (using the drift velocity 
vector at the starting point of a step), the end point of a step 
can be calculated using a (second-order) Runge-Kutta-Fehlberg method. 
This feature can be activated using
\begin{lstlisting}
void EnableRKFSteps(const bool on = true); 
\end{lstlisting}
The average velocity $\mathbf{v}$ over a step $\Delta{t}$ is then 
calculated using
\begin{equation*}
\mathbf{v} = \sum\limits_{k=0}^{2}C_{k}\mathbf{v}_{d}\left(\mathbf{x}_{k}\right), \quad 
\mathbf{x}_{1} = \mathbf{x}_{0} + \Delta{t}\beta_{1,0}\mathbf{v}_{d}\left(\mathbf{x}_{0}\right), \quad 
\mathbf{x}_{2} = \mathbf{x}_{0} + \Delta{t}\left(\beta_{2,0}\mathbf{v}_{d}\left(\mathbf{x}_{0}\right) + \beta_{2,1}\mathbf{v}_{d}\left(\mathbf{x}_{1}\right)\right),
\end{equation*}
where $\mathbf{x}_{0}$ is the starting point of the step. The values of the 
coefficients $C_{k}$ and $\beta_{k,i}$ are given in 
Table~\ref{Tab:CoefficientsRKF}.

If the electric field or drift speed is zero, the algorithm switches 
to diffusion-only steps based on the low-field mobility.
%drawing Gaussian-distributed random steps in all three directions.
%In case of fixed time steps $\Delta{t}$, the $\sigma$ of the 
%Gaussian distribution is given by $\sqrt{2D\Delta{t}}$.
%In case of fixed distance steps $\Delta{s}$, we take $\sigma = \Delta{s}$ 
%and calculate the corresponding time step using 
%$\Delta{t} = \sigma^2/\left((2D\right)$. In case of ``collision'' steps, 
%we calculate $\sigma$ using $\sigma = n D / v_{\text{th}}$, where 
%$v_{\text{th}} = \sqrt{2 k_{\text{B}}T / m}$ is the thermal velocity, 
%and subsequently determine the corresponding time step as in the case of 
%fixed distance steps.
%Note that the diffusion coefficient $D$ used here has dimensions 
%$\text{cm}^2/\text{ns}$.

Drift line calculations are started using 
\begin{lstlisting}
bool DriftElectron(const double x0, const double y0, const double z0, const double t0);
bool DriftHole(const double x0, const double y0, const double z0, const double t0);
bool DriftIon(const double x0, const double y0, const double z0, const double t0);
\end{lstlisting}
\begin{description}
  \item[x0, y0, z0, t0] initial position and time
\end{description}
The trajectory can be retrieved using
\begin{lstlisting} 
unsigned int GetNumberOfDriftLinePoints() const;
void GetDriftLinePoint(const int i, double& x, double& y, double& z, double& t);
\end{lstlisting}

The calculation of an avalanche initiated by an electron, 
a hole or an electron-hole pair is done using
\begin{lstlisting}
bool AvalancheElectron(const double x0, const double y0, const double z0,
                       const double t0, const bool hole = false);
bool AvalancheHole(const double x0, const double y0, const double z0,
                   const double t0, const bool electron = false);
bool AvalancheElectronHole(const double x0, const double y0, const double z0, 
                           const double t0);
\end{lstlisting}
The flags \texttt{hole} and \texttt{electron} specify whether the  
drift and multiplication of the holes/ions (electrons) created in the 
avalanche should be simulated. 
%In case of gas-based detectors, only \texttt{AvalancheElectron} with 
%\texttt{hole = false} is meaningful. 

The starting and endpoints of electrons in the avalanche can be 
retrieved using
\begin{lstlisting}
unsigned int GetNumberOfElectronEndpoints() const;
void GetElectronEndpoint(const unsigned int i, 
                         double& x0, double& y0, double& z0, double& t0,
                         double& x1, double& y1, double& z1, double& t1, int& status) const;
\end{lstlisting}
\begin{description}
  \item[i] index of the electron
  \item[x0, y0, z0, t0] initial position and time of the electron
  \item[x1, y1, z1, t1] final position and time of the electron
  \item[status] status code indicating why the tracking of the electron was stopped.  
\end{description}
Analogous functions are available for holes and ions.

The functions
\begin{lstlisting}
void EnableMagneticField();
\end{lstlisting}
instructs the class to consider not only the electric but also the magnetic field
in the evaluation of the transport parameters.  
By default, magnetic fields are not taken into account.

For debugging purposes, attachment and diffusion can be switched off using
\begin{lstlisting}
void DisableAttachment();
void DisableDiffusion();
\end{lstlisting}

A time interval can be set using
\begin{lstlisting}
void SetTimeWindow(const double t0, const double t1);
\end{lstlisting}
\begin{description}
  \item[t0] lower limit of the time window
  \item[t1] upper limit of the time window 
\end{description}
If a time window is set, 
only charge carriers with a time coordinate \(t \in \left[t_{0}, t_{1}\right]\) 
are tracked. If the time coordinate of a particle crosses the upper limit, 
it is stopped and assigned the status code -17.
Slicing the calculation into time steps can be useful for instance 
for making a movie of the avalanche evolution 
or for calculations involving space charge. 
Another useful function for that purpose is
\begin{lstlisting}
bool ResumeAvalanche(const bool electron = true, const bool hole = true);
\end{lstlisting}
\begin{description}
  \item[electron,hole] flags to switch off the electron (hole) component of the avalanche
\end{description}
which instructs \texttt{AvalancheMC} to continue the avalanche simulation 
from the most recent set of end points. 

The time window can be removed using
\begin{lstlisting}
void UnsetTimeWindow();
\end{lstlisting}

Using the function 
\begin{lstlisting}
void EnableProjectedPathIntegration(const bool on = true);
\end{lstlisting} 
one can request the Townsend and attachment coefficients to 
be projected onto the local drift velocity vector when integrating 
them over drift path segments.
By default, this feature is switched off.
%It ensures that the path length integral does not depend on the step size.
The function 
\begin{lstlisting}
void EnableAvalancheSizeLimit(const unsigned int size);
\end{lstlisting}
sets an upper limit to the number of electrons in an avalanche 
can be imposed.

\section{Microscopic tracking}\label{Sec:MicroscopicTracking}
In the microscopic tracking approach -- implemented at present only for electrons --
a particle is followed from collision to collision. As input, it requires 
a table of the collision rates $\tau^{-1}_{i}\left(\epsilon\right)$ for each 
scattering process $i$ as function of the electron energy $\epsilon$. 
For gases, these data are provided by the class \texttt{MediumMagboltz}.
Between collisions, an electron is traced on a classical vacuum trajectory 
according to the local electric (and optionally magnetic) field. 
The duration $\Delta{t}$ of a free-flight step is controlled by the 
total collision rate 
$\tau^{-1}\left(\epsilon\right) = 
\sum_{i}\tau^{-1}_{i}\left(\epsilon\right)$.
The sampling of $\Delta{t}$ of a free-flight step is done using the 
``null-collision'' method \cite{Skullerud1968}, which accounts for 
the change in electron energy during the step. 
After the step, the energy, direction, and position of the electron 
are updated and the scattering process to take place is sampled based 
on the relative collision rates at the new energy $\epsilon'$.
The energy and direction of the electron are subsequently updated 
according to the type of collision.

In Garfield++, the microscopic tracking method is implemented in the class 
\texttt{AvalancheMicroscopic}. A calculation is started by means of
\begin{lstlisting}
void AvalancheElectron(const double x0, const double y0, const double z0,
                       const double t0, const double e0,
                       const double dx0 = 0., const double dy0 = 0., const double dz0 = 0.);
\end{lstlisting}
\begin{description}
  \item[x0, y0, z0, t0] initial position and time
  \item[e0] initial energy (eV)
  \item[dx0, dy0, dz0] initial direction 
\end{description}
If the norm of the direction vector is zero, 
the initial direction is randomized.

After the calculation is finished, the number of electrons 
(\texttt{ne}) and ions (\texttt{ni})  
produced in the avalanche can be retrieved using
\begin{lstlisting}
void GetAvalancheSize(int& ne, int& ni);
\end{lstlisting}
Information about the ``history'' of each avalanche electron can be 
retrieved by
\begin{lstlisting}
unsigned int GetNumberOfElectronEndpoints() const;
void GetElectronEndpoint(const unsigned int i, 
                         double& x0, double& y0, double& z0, double& t0, double& e0,
                         double& x1, double& y1, double& z1, double& t1, double& e1,
                         int& status); 
\end{lstlisting}
\begin{description}
  \item[i] index of the electron
  \item[x0, y0, z0, t0, e0] initial position, time and energy of the electron
  \item[x1, y1, z1, t1, e1] final position, time and energy of the electron
  \item[status] status code indicating why the tracking of the electron was stopped.  
\end{description}
A list of status codes is given in Table~\ref{Tab:DriftLineStatusCodes}.

The function
\begin{lstlisting}
bool DriftElectron(const double x0, const double y0, const double z0, const double t0,
                   const double e0, const double dx0, const double dy0, const double dz0);
\end{lstlisting}
traces only the initial electron but not the secondaries 
produced along its drift path 
(the input parameters are the same as for \texttt{AvalancheElectron}).

\begin{table}
  \centering
  \begin{tabular}{r l}
    \toprule
    status code &  meaning\\
    \midrule
     -1 & particle left the drift area        \\
     -3 & calculation abandoned (error, should not happen) \\
     -5 & particle not inside a drift medium  \\
     -7 & attachment                          \\
     -8 & sharp kink (only for RKF)           \\
    -16 & energy below transport cut          \\
    -17 & outside the time window             \\
    \bottomrule
  \end{tabular}
  \caption{Status codes for the termination of drift lines.}
  \label{Tab:DriftLineStatusCodes}
\end{table}

The electron energy distribution can be extracted in the following way:
\begin{lstlisting}
AvalancheMicroscopic aval;
// Make a histogram (100 bins between 0 and 100 eV).
TH1F hEnergy("hEnergy", "Electron energy", 100, 0., 100.);
// Pass the histogram to the avalanche class.
aval.EnableElectronEnergyHistogramming(&hEnergy);
\end{lstlisting} 
After each collision, 
the histogram is filled with the current electron energy. 

The effect of magnetic fields can be included 
in the stepping algorithm using the function
\begin{lstlisting}
void EnableMagneticField();
\end{lstlisting}
By default, magnetic fields are not taken into account in the calculation.

Using 
\begin{lstlisting}
void EnableAvalancheSizeLimit(const unsigned int size);
\end{lstlisting}
the size of an electron avalanche can be limited. 
After the avalanche has reached the specified max. size, 
no further secondaries are added to the stack of electrons to be transported.  

Like in \texttt{AvalancheMC} a time window can be set/unset using
\begin{lstlisting}
void SetTimeWindow(const double t0, const double t1);
void UnsetTimeWindow();
\end{lstlisting}

An energy threshold for transporting electrons can be applied using 
\begin{lstlisting}
void SetElectronTransportCut(const double cut);
\end{lstlisting}
\begin{description}
  \item[cut] energy threshold (in eV)
\end{description}
The tracking of an electron is aborted if its energy falls below the 
transport cut. This option can be useful for \(\delta\) electron studies in 
order to stop the calculation once the energy of an electron 
is below the ionization potential of the gas. 
The transport cut can be removed by setting the threshold to a negative value.
By default, no cut is applied.

In order to extract information from the avalanche on a collision-by-collision basis, 
so-called ``user handles'' are available. 
\begin{lstlisting}
 void SetUserHandleStep(void (*f)(double x, double y, double z,
                                  double t, double e,
                                  double dx, double dy, double dz,
                                  bool hole));
void UnsetUserHandleStep();
void SetUserHandleCollision(void (*f)(double x, double y, double z, double     t,
                                      int type, int level, Medium* m,
                                      double e0, double e1,
                                      double dx0, double dy0, double dz0,
                                      double dx1, double dy1, double dz1));
void UnsetUserHandleCollision();
void SetUserHandleAttachment(void (*f)(double x, double y, double z,
                                       double t,
                                       int type, int level, Medium* m));
void UnsetUserHandleAttachment();
void SetUserHandleInelastic(void (*f)(double x, double y, double z,
                                      double t,
                                      int type, int level, Medium* m));
void UnsetUserHandleInelastic();
void SetUserHandleIonisation(void (*f)(double x, double y, double z,
                                       double t,
                                       int type, int level, Medium* m));
void UnsetUserHandleIonisation();
\end{lstlisting}
The function specified in \texttt{SetUserHandleStep} is called 
prior to each free-flight step. 
The parameters passed to this function are 
\begin{description}
  \item[x, y, z, t] 
  position and time, 
  \item[e]
  energy before the step
  \item[dx, dy, dz] 
  direction,
  \item[hole]
  flag indicating whether the particle is an electron or a hole.
\end{description}  
The ``user handle'' function set via \texttt{SetUserHandleCollision} 
is called every time a real collision (as opposed to a null collision) occurs.
The ``user handle'' functions for attachment, ionisation, and inelastic collisions 
are called each time a collision of the respective type occurs.  
In this context, inelastic collisions also include excitations. 
The parameters passed to these functions are 
\begin{description}
  \item[x, y, z, t]
  the location and time of the collision, 
  \item[type]
  the type of collision (see Table~\ref{Tab:ElectronCollisionType}), 
  \item[level]
  the index of the cross-section term (as obtained from the \texttt{Medium}),
  \item[m]
   a pointer to the current \texttt{Medium}. 
\end{description}
In the function set using \texttt{SetUserHandleCollision}, the energy 
and the direction vector before and after the collision are available 
in addition. 

In the following example we want to retrieve  
all excitations happening in the avalanche.
\begin{lstlisting}
void userHandle(double x, double y, double z, double t,
                int type, int level, Medium* m) {

  // Check if the collision is an excitation.
  if (type != 4) return;
  // Do something (e. g. fill a histogram, simulate the emission of a VUV photon) 
  ...
} 

int main(int argc, char* argv[]) {

  // Setup gas, geometry, and field
  ...
  AvalancheMicroscopic* aval = new AvalancheMicroscopic();
  ...
  aval->SetUserHandleInelastic(userHandle);
  double x0 = 0., y0 = 0., z0 = 0., t0 = 0.;
  double e0 = 1.;
  aval->AvalancheElectron(x0, y0, z0, t0, e0, 0., 0., 0.); 
  ...

}

\end{lstlisting}
 
\section{Visualizing drift lines}

For plotting drift lines and tracks the class \texttt{ViewDrift} can be used. 
After attaching a \texttt{ViewDrift} object to a transport class, 
\textit{e.~g.} using
\begin{lstlisting}
void AvalancheMicroscopic::EnablePlotting(ViewDrift* view);
void AvalancheMC::EnablePlotting(ViewDrift* view);
void DriftLineRKF::EnablePlotting(ViewDrift* view);
void Track::EnablePlotting(ViewDrift* view);
\end{lstlisting}
\texttt{ViewDrift} stores the trajectories which are calculated by the 
transport class. 
The drawing of the trajectories is triggered by the function
\begin{lstlisting}
void ViewDrift::Plot();
\end{lstlisting}

In case of \texttt{AvalancheMicroscopic}, it is usually not advisable to 
plot every single collision. The number of collisions 
to be skipped for plotting can be set using
\begin{lstlisting}
void AvalancheMicroscopic::SetCollisionSteps(const int n);
\end{lstlisting}
\begin{description}
  \item[n] number of collisions to be skipped
\end{description}
Note that this setting does not affect the transport of the electron as such, 
the electron is always tracked rigorously through single collisions.

\section{Visualizing isochrons}

For drift chambers, it is useful to determine the contours of equal drift time to a wire.
These so-called isochrons can be calculated and drawn 
using the class \texttt{ViewIsochrons}.
The component or sensor from which to retrieve the field is set using
\begin{lstlisting}
void SetComponent(Component* c);
\end{lstlisting}
or
\begin{lstlisting}
void SetSensor(Sensor* s);
\end{lstlisting}

The function
\begin{lstlisting}
void DriftElectrons(const bool positive = false);
\end{lstlisting}
instructs the class to compute isochrons using electron drift lines.
If the flag \texttt{positive} is set to \texttt{true}, the electrons 
are drifted with positive charge,
which is useful for calculating isochrons of wires that attract electrons.

By calling
\begin{lstlisting}
void DriftIons(const bool negative = false);
\end{lstlisting}
one requests drift lines of (positive or negative) ions.

The calculation of the drift lines and equal time contours and their visualization is 
done by the function 
\begin{lstlisting}
void PlotIsochrons(const double tstep,
    const std::vector<std::array<double, 3> >& points, const bool reverse = false,
    const bool colour = false, const bool markers = false, const bool plotDriftLines = true);
\end{lstlisting}
\begin{description}
  \item[tstep] time interval between isochron lines,
  \item[points] list of starting points from which to simulate drift lines,
  \item[reverse] flag to measure the drift time from the end points of the drift lines (\texttt{true}) or from the starting points (\texttt{false}),
  \item[colour] requests drawing the contour lines using the currently active colour palette,
  \item[markers] flag to draw markers at the points on the isochrons (\texttt{true}) or draw the isochron as lines (\texttt{false}), 
  \item[plotDriftLines] requests plotting of the drift lines together with the isochrons. 
\end{description} 
The calculation of the drift lines is done using \texttt{DriftLineRKF}.

The appearance of the isochrons is affected by a number of additional parameters.
By default, the algorithm tries to order the points at equal time such that 
the isochrons appear as reasonably smooth lines. 
This sorting step can be switched off or on using
\begin{lstlisting}
void EnableSorting(const bool on = true);
\end{lstlisting} 
When an isochron appears to be more or less circular, its points 
are ordered by increasing angle with respect to the centre of gravity. 
If the isochron, on the other hand, seems to be more or less linear, 
the points are ordered along the longest principal axis of the distribution.
Whether the set is circular or linear is decided by computing the RMS 
in the two principal axes of the point distribution. 
If the ratio of these two numbers exceeds a threshold
then the isochron is assumed to be linear, otherwise circular.
This parameter (which defaults to 3) can be set using
\begin{lstlisting}
void SetAspectRatioSwitch(const double ar);
\end{lstlisting}
Isochrons that appear to be circular are closed if the largest distance 
between two points does not exceed a certain fraction of the 
total length of the isochron. 
This threshold (initially set to 0.2) can be modified using
\begin{lstlisting}
void SetLoopThreshold(const double thr);
\end{lstlisting}

Points on an isochron are only joined if they are less than a 
certain fraction away from each other on the screen. 
Points that can not be connected are shown by a marker. 
This fraction (initial value: 0.2) can be set using
\begin{lstlisting}
void SetConnectionThreshold(const double thr);
\end{lstlisting}
By default, points on an isochron are also not joined if their connecting 
line crosses a drift line.
This feature can be switched on or off using
\begin{lstlisting}
void CheckCrossings(const bool on = true);
\end{lstlisting}


